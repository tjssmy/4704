\documentclass[11pt,a4paper]{article}
\usepackage{amsmath}
\usepackage{amsfonts}
\usepackage{amssymb}
\usepackage{graphicx}
\usepackage{geometry}
\usepackage{float}
\usepackage{listings}
\usepackage{xcolor}
\usepackage{hyperref}

\geometry{margin=1in}

\hypersetup{
    colorlinks=true,
    linkcolor=blue,
    filecolor=magenta,      
    urlcolor=cyan,
}

\title{Electromagnetic Scattering from a Perfectly Conducting Cylinder: \\ A Modern Implementation}
\author{Generated by Gemini}
\date{\today}

\begin{document}

\maketitle

\begin{abstract}
This document details the physics, mathematics, and numerical implementation for solving the two-dimensional electromagnetic scattering of a plane wave from a perfectly conducting (PEC) cylinder. The solution is derived using the classical method of eigenfunction expansion in cylindrical coordinates. We present a clear, modern Python implementation that visualizes the incident, scattered, and total electric fields.
\end{abstract}

\section{Physical Problem}

We analyze the scattering of a time-harmonic, $E_z$-polarized plane wave incident upon an infinitely long, perfectly conducting cylinder of radius $b$.

\subsection{Key Components}
\begin{itemize}
    \item \textbf{Incident Wave}: A plane wave with electric field polarized along the z-axis ($E_z$), propagating in the $+x$ direction. Its mathematical form is $E_z^{\text{inc}} = E_0 e^{i(kx - \omega t)}$.
    \item \textbf{Scatterer}: An infinitely long cylinder of radius $b$, made of a perfect electrical conductor (PEC).
    \item \textbf{Medium}: The cylinder is situated in a homogeneous, isotropic, non-magnetic free space.
    \item \textbf{Boundary Condition}: The tangential component of the total electric field must be zero on the surface of a PEC. For this problem, this simplifies to $E_z^{\text{total}}(r=b, \phi) = 0$.
\end{itemize}

\section{Mathematical Formulation}

The problem is governed by the time-harmonic Maxwell's equations, which for this 2D TE (Transverse Electric) case, reduce to the scalar Helmholtz equation for $E_z$:
\begin{equation}
    (\nabla^2 + k^2) E_z(r, \phi) = 0
\end{equation}
where $k = 2\pi/\lambda$ is the wave number.

\subsection{Eigenfunction Expansion in Cylindrical Coordinates}
We express the total field as a superposition of the incident and scattered fields:
\begin{equation}
    E_z^{\text{total}} = E_z^{\text{inc}} + E_z^{\text{scat}}
\end{equation}

\subsubsection{Incident Field Expansion}
The incident plane wave $E_z^{\text{inc}} = E_0 e^{ikx} = E_0 e^{ikr\cos\phi}$ is expanded in a series of cylindrical basis functions using the \textbf{Jacobi-Anger identity}:
\begin{equation}
    E_z^{\text{inc}}(r, \phi) = E_0 \sum_{n=-\infty}^{\infty} i^n J_n(kr) e^{in\phi}
    \label{eq:incident}
\end{equation}
where $J_n$ are Bessel functions of the first kind, representing standing waves.

\subsubsection{Scattered Field Expansion}
The scattered field must represent waves propagating radially outward from the cylinder. This is captured by an expansion using \textbf{Hankel functions of the first kind}, $H_n^{(1)}$:
\begin{equation}
    E_z^{\text{scat}}(r, \phi) = \sum_{n=-\infty}^{\infty} a_n H_n^{(1)}(kr) e^{in\phi}
    \label{eq:scattered}
\end{equation}
where $a_n$ are the unknown scattering coefficients.

\subsection{Determining the Scattering Coefficients}
We apply the PEC boundary condition at $r=b$:
\begin{equation}
    E_z^{\text{inc}}(b, \phi) + E_z^{\text{scat}}(b, \phi) = 0
\end{equation}
Substituting the expansions from Eq. \ref{eq:incident} and \ref{eq:scattered}:
\begin{equation}
    \sum_{n=-\infty}^{\infty} \left[ E_0 i^n J_n(kb) + a_n H_n^{(1)}(kb) \right] e^{in\phi} = 0
\end{equation}
For this equation to hold for all $\phi$, the term in the brackets must be zero for each mode $n$. Solving for $a_n$ gives:
\begin{equation}
    a_n = -E_0 i^n \frac{J_n(kb)}{H_n^{(1)}(kb)}
    \label{eq:coeffs}
\end{equation}
These coefficients uniquely determine the scattered field.

\section{Numerical Implementation}

The solution is implemented in Python using the `numpy` and `scipy` libraries.

\subsection{Core Algorithm}
\begin{enumerate}
    \item \textbf{Initialization}: Define physical parameters: cylinder radius $b$, wavelength $\lambda$, and incident field amplitude $E_0$.
    \item \textbf{Series Truncation}: The infinite series must be truncated. A common rule of thumb is to sum up to $N \approx |kb| + 10$ to ensure convergence.
    \item \textbf{Coefficient Calculation}: Compute the scattering coefficients $a_n$ for each mode $n$ from $-N$ to $N$ using Eq. \ref{eq:coeffs}. This is done once and stored.
    \item \textbf{Grid Generation}: Create a 2D Cartesian grid and convert it to cylindrical coordinates $(r, \phi)$.
    \item \textbf{Field Calculation}: For each point on the grid, compute $E_z^{\text{inc}}$ and $E_z^{\text{scat}}$ by summing the series.
    \item \textbf{Enforce Boundary}: Set the total field $E_z^{\text{total}}$ to zero for all points inside or on the cylinder ($r \le b$).
    \item \textbf{Visualization}: Plot the real part of the incident, scattered, and total fields using `matplotlib`.
\end{enumerate}

\subsection{Python Code Snippet}
The core of the field calculation is shown below.
\begin{lstlisting}[language=Python, caption=Field Calculation Loop]
# Inside the get_fields method of the solver class
Ez_inc = np.zeros_like(r, dtype=complex)
Ez_scat = np.zeros_like(r, dtype=complex)

for n, an in self.coefficients.items():
    # Incident field using Jacobi-Anger expansion
    jn_kr = jv(n, self.k * r)
    Ez_inc += self.E0 * (1j)**n * jn_kr * np.exp(1j * n * phi)

    # Scattered field using Hankel functions
    hn_kr = hankel1(n, self.k * r)
    Ez_scat += an * hn_kr * np.exp(1j * n * phi)

# Total field is the sum, but zero inside the cylinder
Ez_total = Ez_inc + Ez_scat
Ez_total[r <= self.b] = 0
\end{lstlisting}

\section{Results and Visualization}
The program generates a three-panel figure illustrating the different field components.
\begin{figure}[H]
    \centering
    \includegraphics[width=\textwidth]{em_scattering_cylinder_gemini.png}
    \caption{Visualization of the real part of the incident, scattered, and total electric fields for a plane wave incident on a PEC cylinder. The shadow region behind the cylinder and the interference patterns are clearly visible.}
    \label{fig:fields}
\end{figure}

\subsection{Interpretation of the Plots}
\begin{itemize}
    \item \textbf{Incident Field}: Shows the undisturbed plane wave fronts.
    \item \textbf{Scattered Field}: Shows the cylindrical waves radiating from the cylinder. The pattern is complex due to the interference of all the modes $n$.
    \item \textbf{Total Field}: Shows the superposition of the two. Key features include:
    \begin{itemize}
        \item A "shadow" region behind the cylinder where the field is weak.
        \item Standing wave patterns in front of the cylinder due to interference between the incident and reflected waves.
        \item The field is exactly zero inside the cylinder, as required.
    \end{itemize}
\end{itemize}

\section{Conclusion}
The eigenfunction expansion method provides a powerful and accurate tool for solving canonical scattering problems. This implementation successfully models the interaction of a plane wave with a conducting cylinder, providing clear visualizations that align with electromagnetic theory. The resulting plots offer valuable physical insight into the phenomena of scattering, diffraction, and interference.

\end{document}
