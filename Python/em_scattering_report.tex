\documentclass[11pt,a4paper]{article}
\usepackage{amsmath}
\usepackage{amsfonts}
\usepackage{amssymb}
\usepackage{graphicx}
\usepackage{geometry}
\usepackage{float}
\usepackage{listings}
\usepackage{xcolor}
\usepackage{physics}
\usepackage{braket}

\geometry{margin=1in}

% Code listing settings
\lstset{
    language=Python,
    basicstyle=\ttfamily\small,
    keywordstyle=\color{blue},
    commentstyle=\color{green!60!black},
    stringstyle=\color{red},
    showstringspaces=false,
    breaklines=true,
    frame=single,
    numbers=left,
    numberstyle=\tiny
}

\title{Electromagnetic Scattering from a Perfectly Conducting Cylinder \\
       Using Eigenfunction Expansion}
\author{Course 4704}
\date{\today}

\begin{document}

\maketitle

\section{Introduction}

This document describes the theoretical foundation, mathematical formulation, and computational implementation of electromagnetic wave scattering from a perfectly conducting circular cylinder. The problem represents a classical example in electromagnetic theory with applications in radar cross-section analysis, antenna design, and optical scattering.

\section{Physical Problem Statement}

\subsection{Geometry and Setup}

We consider a two-dimensional electromagnetic scattering problem with the following configuration:

\begin{itemize}
\item \textbf{Incident Wave}: Plane wave propagating in the $+x$ direction
\item \textbf{Polarization}: $E_z$ polarization (TE mode, electric field parallel to cylinder axis)
\item \textbf{Scatterer}: Perfectly conducting circular cylinder of radius $b$ centered at origin
\item \textbf{Coordinate System}: Cylindrical coordinates $(r, \phi, z)$
\item \textbf{Boundary Condition}: $E_z = 0$ at $r = b$ (perfect conductor)
\end{itemize}

\subsection{Physical Assumptions}

\begin{enumerate}
\item \textbf{Time-harmonic fields}: $e^{-i\omega t}$ time dependence
\item \textbf{Two-dimensional problem}: No variation in $z$-direction ($\partial/\partial z = 0$)
\item \textbf{Linear medium}: Free space with $\epsilon = \epsilon_0$, $\mu = \mu_0$
\item \textbf{Perfect conductor}: Infinite conductivity, zero field penetration
\end{enumerate}

\section{Mathematical Formulation}

\subsection{Maxwell's Equations}

For time-harmonic fields with $E_z$ polarization, Maxwell's equations reduce to the scalar wave equation:

\begin{equation}
\nabla^2 E_z + k^2 E_z = 0 \label{eq:wave_equation}
\end{equation}

where $k = \omega/c = 2\pi/\lambda$ is the wave number.

\subsection{Cylindrical Wave Equation}

In cylindrical coordinates, the Laplacian becomes:
\begin{equation}
\nabla^2 E_z = \frac{1}{r}\frac{\partial}{\partial r}\left(r \frac{\partial E_z}{\partial r}\right) + \frac{1}{r^2}\frac{\partial^2 E_z}{\partial \phi^2} \label{eq:laplacian_cyl}
\end{equation}

The complete wave equation is:
\begin{equation}
\frac{1}{r}\frac{\partial}{\partial r}\left(r \frac{\partial E_z}{\partial r}\right) + \frac{1}{r^2}\frac{\partial^2 E_z}{\partial \phi^2} + k^2 E_z = 0 \label{eq:helmholtz_cyl}
\end{equation}

\subsection{Separation of Variables}

Assuming a solution of the form $E_z(r,\phi) = R(r)\Phi(\phi)$ and separating variables:

\begin{align}
\frac{r}{R}\frac{d}{dr}\left(r \frac{dR}{dr}\right) + k^2 r^2 &= m^2 \label{eq:radial_separated} \\
\frac{d^2\Phi}{d\phi^2} + m^2 \Phi &= 0 \label{eq:angular_separated}
\end{align}

where $m^2$ is the separation constant.

\subsection{Angular Solutions}

For periodic boundary conditions $\Phi(\phi + 2\pi) = \Phi(\phi)$, we require $m$ to be an integer:

\begin{equation}
\Phi_m(\phi) = \begin{cases}
A_m \cos(m\phi) + B_m \sin(m\phi) & \text{for } m > 0 \\
A_0 & \text{for } m = 0
\end{cases} \label{eq:angular_solutions}
\end{equation}

\subsection{Radial Solutions}

The radial equation becomes Bessel's equation:
\begin{equation}
r^2 \frac{d^2R}{dr^2} + r \frac{dR}{dr} + (k^2r^2 - m^2)R = 0 \label{eq:bessel_equation}
\end{equation}

The general solution involves Bessel and Neumann functions:
\begin{equation}
R_m(r) = C_1 J_m(kr) + C_2 Y_m(kr) \label{eq:radial_general}
\end{equation}

For outgoing waves (scattered field), we use Hankel functions of the first kind:
\begin{equation}
H_m^{(1)}(kr) = J_m(kr) + iY_m(kr) \label{eq:hankel_function}
\end{equation}

\section{Eigenfunction Expansion Solution}

\subsection{Incident Field Representation}

The incident plane wave $E_z^{\text{inc}} = E_0 e^{ikx}$ can be expressed using the Jacobi-Anger expansion:

\begin{equation}
e^{ikr\cos\phi} = \sum_{m=-\infty}^{\infty} i^m J_m(kr) e^{im\phi} \label{eq:jacobi_anger}
\end{equation}

Therefore:
\begin{equation}
E_z^{\text{inc}}(r,\phi) = E_0 \sum_{m=-\infty}^{\infty} i^m J_m(kr) e^{im\phi} \label{eq:incident_expansion}
\end{equation}

\subsection{Scattered Field Representation}

The scattered field must satisfy the radiation condition (outgoing waves) and is represented as:

\begin{equation}
E_z^{\text{scat}}(r,\phi) = \sum_{m=-\infty}^{\infty} a_m H_m^{(1)}(kr) e^{im\phi} \label{eq:scattered_expansion}
\end{equation}

where $a_m$ are unknown scattering coefficients.

\subsection{Boundary Conditions}

At the perfectly conducting cylinder surface ($r = b$), the total tangential electric field must vanish:

\begin{equation}
E_z^{\text{total}}(b,\phi) = E_z^{\text{inc}}(b,\phi) + E_z^{\text{scat}}(b,\phi) = 0 \label{eq:boundary_condition}
\end{equation}

Substituting the expansions:
\begin{equation}
\sum_{m=-\infty}^{\infty} \left[E_0 i^m J_m(kb) + a_m H_m^{(1)}(kb)\right] e^{im\phi} = 0 \label{eq:boundary_expanded}
\end{equation}

Since this must hold for all $\phi$, each coefficient must vanish:
\begin{equation}
E_0 i^m J_m(kb) + a_m H_m^{(1)}(kb) = 0 \label{eq:coefficient_equation}
\end{equation}

\subsection{Scattering Coefficients}

Solving for the scattering coefficients:
\begin{equation}
a_m = -E_0 i^m \frac{J_m(kb)}{H_m^{(1)}(kb)} \label{eq:scattering_coefficients}
\end{equation}

\section{Complete Field Solutions}

\subsection{Total Field}

The total field is given by:
\begin{equation}
E_z^{\text{total}}(r,\phi) = \begin{cases}
E_z^{\text{inc}}(r,\phi) + E_z^{\text{scat}}(r,\phi) & \text{for } r > b \\
0 & \text{for } r \leq b
\end{cases} \label{eq:total_field}
\end{equation}

\subsection{Field Components}

\begin{align}
E_z^{\text{inc}}(r,\phi) &= E_0 \sum_{m=-\infty}^{\infty} i^m J_m(kr) e^{im\phi} \label{eq:incident_final} \\
E_z^{\text{scat}}(r,\phi) &= -E_0 \sum_{m=-\infty}^{\infty} i^m \frac{J_m(kb)}{H_m^{(1)}(kb)} H_m^{(1)}(kr) e^{im\phi} \label{eq:scattered_final}
\end{align}

\section{Scattering Cross Section}

\subsection{Optical Theorem}

The scattering cross section can be calculated using the optical theorem:
\begin{equation}
\sigma_{\text{scat}} = \frac{4}{k} \sum_{m=-\infty}^{\infty} |a_m|^2 \label{eq:scattering_cross_section}
\end{equation}

\subsection{Efficiency Factor}

The scattering efficiency is defined as:
\begin{equation}
Q_{\text{scat}} = \frac{\sigma_{\text{scat}}}{2b} \label{eq:efficiency_factor}
\end{equation}

where $2b$ is the geometric cross section.

\section{Computational Implementation}

\subsection{Numerical Considerations}

\begin{enumerate}
\item \textbf{Series Truncation}: The infinite series are truncated at $|m| \leq N$ where $N \approx kb + 10$
\item \textbf{Bessel Function Evaluation}: Use scipy.special functions for numerical stability
\item \textbf{Hankel Functions}: Computed as $H_m^{(1)} = J_m + iY_m$
\item \textbf{Grid Generation}: Uniform Cartesian grid converted to cylindrical coordinates
\end{enumerate}

\subsection{Algorithm Structure}

\begin{lstlisting}[caption=Main Algorithm Structure]
def solve_scattering():
    1. Initialize parameters (k, b, E0)
    2. Calculate scattering coefficients a_m
    3. Generate computational grid
    4. For each grid point (r, phi):
        a. Calculate incident field
        b. Calculate scattered field (if r > b)
        c. Calculate total field
    5. Create visualization plots
    6. Analyze scattering properties
\end{lstlisting}

\subsection{Scattering Coefficient Calculation}

\begin{lstlisting}[caption=Scattering Coefficients]
def calculate_scattering_coefficients(k, b):
    n_terms = max(20, int(2 * k * b + 10))
    coefficients = []
    
    for m in range(-n_terms, n_terms + 1):
        jm_kb = jv(m, k * b)           # J_m(kb)
        hm_kb = hankel1(m, k * b)      # H_m^(1)(kb)
        
        am = -jm_kb / hm_kb            # Scattering coefficient
        coefficients.append(am)
    
    return coefficients
\end{lstlisting}

\subsection{Field Evaluation}

\begin{lstlisting}[caption=Field Calculation]
def calculate_fields(r, phi, k, coefficients):
    E_inc = 0
    E_scat = 0
    
    n_terms = len(coefficients) // 2
    
    for m in range(-n_terms, n_terms + 1):
        # Incident field (Jacobi-Anger expansion)
        coeff_inc = (1j) ** m
        jm_kr = jv(m, k * r)
        E_inc += coeff_inc * jm_kr * exp(1j * m * phi)
        
        # Scattered field (r > b only)
        if r > b:
            am = coefficients[m + n_terms]
            hm_kr = hankel1(m, k * r)
            E_scat += am * hm_kr * exp(1j * m * phi)
    
    return E_inc, E_scat
\end{lstlisting}

\section{Physical Regimes}

\subsection{Rayleigh Regime ($kb \ll 1$)}
\begin{itemize}
\item Small particle limit
\item Dominant dipole scattering ($m = \pm 1$ terms)
\item Cross section $\propto k^4$ (Rayleigh scattering)
\item Forward-backward symmetric pattern
\end{itemize}

\subsection{Resonance Regime ($kb \sim 1$)}
\begin{itemize}
\item Particle size comparable to wavelength
\item Multiple terms contribute significantly
\item Complex interference patterns
\item Resonance peaks in scattering efficiency
\end{itemize}

\subsection{Optical Regime ($kb \gg 1$)}
\begin{itemize}
\item Geometric optics limit
\item Sharp shadow boundary
\item Creeping waves around cylinder
\item Efficiency approaches 2 (extinction paradox)
\end{itemize}

\section{Validation and Results}

\subsection{Analytical Benchmarks}

The implementation can be validated against known analytical results:

\begin{enumerate}
\item \textbf{Small cylinder limit}: Rayleigh scattering formula
\item \textbf{Large cylinder limit}: Geometric optics approximation
\item \textbf{Specific values}: Published scattering cross sections
\end{enumerate}

\subsection{Field Pattern Analysis}

The visualizations reveal key physical phenomena:

\begin{itemize}
\item \textbf{Incident field}: Uniform plane wave pattern
\item \textbf{Scattered field}: Cylindrical wave fronts with complex interference
\item \textbf{Total field}: Standing wave patterns and shadow regions
\item \textbf{Boundary condition}: Zero field inside conductor
\end{itemize}

\section{Applications}

\subsection{Engineering Applications}

\begin{itemize}
\item \textbf{Radar Cross Section}: Stealth technology and target identification
\item \textbf{Antenna Design}: Near-field coupling and radiation patterns
\item \textbf{Optical Scattering}: Particle sizing and characterization
\item \textbf{Wireless Communications}: Propagation around obstacles
\end{itemize}

\subsection{Research Areas}

\begin{itemize}
\item \textbf{Metamaterials}: Cloaking and negative refractive index
\item \textbf{Plasmonics}: Surface plasmon resonances in nanoparticles
\item \textbf{Atmospheric Physics}: Scattering by raindrops and ice crystals
\item \textbf{Biomedical Optics}: Light scattering in tissue diagnostics
\end{itemize}

\section{Conclusion}

The eigenfunction expansion method provides an exact analytical solution to electromagnetic scattering from circular cylinders. The implementation demonstrates:

\begin{enumerate}
\item \textbf{Mathematical rigor}: Exact solution satisfying all boundary conditions
\item \textbf{Physical insight}: Clear visualization of scattering phenomena
\item \textbf{Computational efficiency}: Series convergence and numerical stability
\item \textbf{Practical relevance}: Applications across multiple engineering disciplines
\end{enumerate}

The method serves as a fundamental building block for more complex scattering problems and provides essential understanding of wave-matter interactions in electromagnetic theory.

\section{References}

\begin{enumerate}
\item Harrington, R. F. (2001). \textit{Time-Harmonic Electromagnetic Fields}. Wiley-IEEE Press.
\item Bohren, C. F., \& Huffman, D. R. (2008). \textit{Absorption and Scattering of Light by Small Particles}. Wiley.
\item Van de Hulst, H. C. (1981). \textit{Light Scattering by Small Particles}. Dover Publications.
\item Balanis, C. A. (2012). \textit{Advanced Engineering Electromagnetics}. John Wiley \& Sons.
\end{enumerate}

\end{document}
