\documentclass[11pt,a4paper]{article}
\usepackage{amsmath}
\usepackage{amsfonts}
\usepackage{amssymb}
\usepackage{graphicx}
\usepackage{geometry}
\usepackage{float}
\usepackage{listings}
\usepackage{xcolor}

\geometry{margin=1in}

% Code listing settings
\lstset{
    language=Python,
    basicstyle=\ttfamily\small,
    keywordstyle=\color{blue},
    commentstyle=\color{green!60!black},
    stringstyle=\color{red},
    showstringspaces=false,
    breaklines=true,
    frame=single,
    numbers=left,
    numberstyle=\tiny
}

\title{One-Dimensional Laplace Equation Solution \\
       Using Finite Difference Method}
\author{Course 4704}
\date{\today}

\begin{document}

\maketitle

\section{Problem Statement}

We seek to solve the one-dimensional Laplace equation for the function $u(x)$ on the domain $x \in [0, L]$ with mixed boundary conditions:

\begin{align}
\frac{d^2u}{dx^2} &= 0 \quad \text{for } x \in (0, L) \label{eq:laplace} \\
\frac{du}{dx}\bigg|_{x=0} &= 2 \quad \text{(Neumann boundary condition)} \label{eq:bc1} \\
u(L) &= 0 \quad \text{(Dirichlet boundary condition)} \label{eq:bc2}
\end{align}

\section{Analytical Solution}

The general solution to the one-dimensional Laplace equation (\ref{eq:laplace}) is:
\begin{equation}
u(x) = Ax + B
\end{equation}

where $A$ and $B$ are constants to be determined from the boundary conditions.

Applying the Neumann boundary condition (\ref{eq:bc1}):
\begin{equation}
\frac{du}{dx} = A = 2
\end{equation}

Therefore, $A = 2$.

Applying the Dirichlet boundary condition (\ref{eq:bc2}):
\begin{equation}
u(L) = 2L + B = 0
\end{equation}

Therefore, $B = -2L$.

The analytical solution is:
\begin{equation}
u(x) = 2x - 2L = 2(x - L) \label{eq:analytical}
\end{equation}

We can verify this solution satisfies both boundary conditions:
\begin{align}
\frac{du}{dx}\bigg|_{x=0} &= 2 \quad \checkmark \\
u(L) &= 2(L - L) = 0 \quad \checkmark
\end{align}

\section{Finite Difference Method}

\subsection{Grid Discretization}

We discretize the domain $[0, L]$ into $N$ equally spaced grid points:
\begin{equation}
x_i = i \cdot \Delta x, \quad i = 0, 1, 2, \ldots, N-1
\end{equation}

where $\Delta x = \frac{L}{N-1}$ is the grid spacing.

\subsection{Finite Difference Approximations}

For interior points ($i = 1, 2, \ldots, N-2$), we use the central difference approximation for the second derivative:
\begin{equation}
\frac{d^2u}{dx^2}\bigg|_{x=x_i} \approx \frac{u_{i+1} - 2u_i + u_{i-1}}{(\Delta x)^2} = 0
\end{equation}

This gives us the discretized equation:
\begin{equation}
u_{i-1} - 2u_i + u_{i+1} = 0 \label{eq:interior}
\end{equation}

\subsection{Boundary Condition Implementation}

\subsubsection{Neumann Boundary Condition at $x = 0$}

At the left boundary ($i = 0$), we implement the Neumann condition $\frac{du}{dx}|_{x=0} = 2$ using a forward difference approximation:
\begin{equation}
\frac{du}{dx}\bigg|_{x=0} \approx \frac{u_1 - u_0}{\Delta x} = 2
\end{equation}

Rearranging:
\begin{equation}
-u_0 + u_1 = 2\Delta x \label{eq:neumann}
\end{equation}

\subsubsection{Dirichlet Boundary Condition at $x = L$}

At the right boundary ($i = N-1$), we simply impose:
\begin{equation}
u_{N-1} = 0 \label{eq:dirichlet}
\end{equation}

\subsection{Linear System Formation}

The finite difference equations (\ref{eq:interior}), (\ref{eq:neumann}), and (\ref{eq:dirichlet}) form a system of linear equations:
\begin{equation}
\mathbf{A}\mathbf{u} = \mathbf{b}
\end{equation}

where $\mathbf{u} = [u_0, u_1, \ldots, u_{N-1}]^T$ is the solution vector.

The coefficient matrix $\mathbf{A}$ and right-hand side vector $\mathbf{b}$ are constructed as follows:

\begin{itemize}
\item \textbf{Row 0} (Neumann BC): $A_{0,0} = -1$, $A_{0,1} = 1$, $b_0 = 2\Delta x$
\item \textbf{Rows 1 to $N-2$} (Interior points): $A_{i,i-1} = 1$, $A_{i,i} = -2$, $A_{i,i+1} = 1$, $b_i = 0$
\item \textbf{Row $N-1$} (Dirichlet BC): $A_{N-1,N-1} = 1$, $b_{N-1} = 0$
\end{itemize}

\section{Implementation Details}

The solution is implemented in Python using the following key steps:

\begin{enumerate}
\item Create a uniform grid with $N$ points
\item Construct the coefficient matrix $\mathbf{A}$ and right-hand side vector $\mathbf{b}$
\item Solve the linear system using NumPy's \texttt{linalg.solve()} function
\item Compare with the analytical solution for validation
\end{enumerate}

\section{Error Analysis}

The finite difference method introduces discretization error that depends on the grid spacing $\Delta x$. For the second-order central difference scheme used for interior points, the truncation error is $O((\Delta x)^2)$.

The forward difference approximation for the Neumann boundary condition introduces a first-order error $O(\Delta x)$, which can limit the overall accuracy of the scheme.

\section{Results}

The numerical solution converges to the analytical solution as the grid is refined. Key observations:

\begin{itemize}
\item The method accurately captures the linear nature of the solution
\item Boundary conditions are satisfied to machine precision
\item The error decreases as the number of grid points increases
\item The solution exhibits the expected linear profile: $u(x) = 2(x - L)$
\end{itemize}

\section{Conclusion}

The finite difference method provides an effective approach for solving the one-dimensional Laplace equation with mixed boundary conditions. The implementation successfully handles both Neumann and Dirichlet boundary conditions, producing results that agree well with the analytical solution.

The method can be easily extended to:
\begin{itemize}
\item Non-uniform grids
\item Higher-order finite difference schemes
\item Time-dependent problems
\item Multi-dimensional domains
\end{itemize}

\section{Code Listing}

The complete Python implementation is provided in the file \texttt{laplace\_1d\_solver.py}, which includes:
\begin{itemize}
\item Grid generation and matrix assembly
\item Linear system solution
\item Analytical solution comparison
\item Visualization and error analysis
\item Boundary condition verification
\end{itemize}

\end{document}
